%%
%% This is file `template-8d.tex',
%% generated with the docstrip utility.
%%
%% The original source files were:
%%
%% template.raw  (with options: `8d')
%% 
%% Template for the LaTeX class aipproc.
%% 
%% (C) 1998,2000,2001 American Institute of Physics and Frank Mittelbach
%% All rights reserved
%% 
%%
%% $Id: template.raw,v 1.12 2005/07/06 19:22:14 frank Exp $
%%

%%%%%%%%%%%%%%%%%%%%%%%%%%%%%%%%%%%%%%%%%%%%
%% Please remove the next line of code if you
%% are satisfied that your installation is
%% complete and working.
%%
%% It is only there to help you in detecting
%% potential problems.
%%%%%%%%%%%%%%%%%%%%%%%%%%%%%%%%%%%%%%%%%%%%

\input{aipcheck}

%%%%%%%%%%%%%%%%%%%%%%%%%%%%%%%%%%%%%%%%%%%%
%% SELECT THE LAYOUT
%%
%% The class supports further options.
%% See aipguide.pdf for details.
%%
%%%%%%%%%%%%%%%%%%%%%%%%%%%%%%%%%%%%%%%%%%%%

\documentclass[
    ,final            % use final for the camera ready runs
%%  ,draft            % use draft while you are working on the paper
%%  ,numberedheadings % uncomment this option for numbered sections
%%  ,                 % add further options here if necessary
  ]
  {aipproc}

\layoutstyle{8x11double}

%%%%%%%%%%%%%%%%%%%%%%%%%%%%%%%%%%%%%%%%%%%%
%% FRONTMATTER
%%%%%%%%%%%%%%%%%%%%%%%%%%%%%%%%%%%%%%%%%%%%

\begin{document}

\title{Fixed Size Least Squares Support Vector Machines: Scalable Implementation for Large Scale Predictive Models}

\classification{<Replace this text with PACS numbers; choose from this list:
                \texttt{http://www.aip..org/pacs/index.html}>}
\keywords      {FS-LSSVM, Big Data, Large Scale Models, Non-linear SVM}

\author{Mandar Chandorkar}{
  address={ESAT-STADIUS KU Leuven \\ Kasteelpark Arenberg 10, \\ B-3001 Leuven, Belgium}
}

\author{Oliver Lauwers}{
  address={<oliver.lauwers,raghvendra.mall,johan.suykens,bart.demoor>@esat.kuleuven.be},altaddress={ESAT-STADIUS KU Leuven \\ Kasteelpark Arenberg 10, \\ B-3001 Leuven, Belgium}}

\author{Raghvendra Mall}{
  address={<oliver.lauwers,raghvendra.mall,johan.suykens,bart.demoor>@esat.kuleuven.be},altaddress={ESAT-STADIUS KU Leuven \\ Kasteelpark Arenberg 10, \\ B-3001 Leuven, Belgium}}% additional visiting address


\author{Johan A.K Suykens}{
  address={<oliver.lauwers,raghvendra.mall,johan.suykens,bart.demoor>@esat.kuleuven.be},altaddress={ESAT-STADIUS KU Leuven \\ Kasteelpark Arenberg 10, \\ B-3001 Leuven, Belgium} % additional visiting address
}

\author{Bart De Moor}{
  address={<oliver.lauwers,raghvendra.mall,johan.suykens,bart.demoor>@esat.kuleuven.be},altaddress={ESAT-STADIUS KU Leuven \\ Kasteelpark Arenberg 10, \\ B-3001 Leuven, Belgium} % additional visiting address
}



\begin{abstract}
 We propose \textit{Bayes Learn}, a flexible and modular \textit{Scala} based library for the implementing and tuning large scale supervised kernel based models. The framework consists of a set of modules for (gradient and gradient free) optimization, model representation, kernel functions, evaluation and probabilistic graphical models.
 
 A kernel based \emph{Fixed Size Least Squares Support Vector Machine} (FS LS-SVM) model is implemented using the proposed framework, while heavily leveraging the parallel computing capabilities of \textit{Apache Spark}. Global optimization routines like \emph{Coupled Simulated Annealing} (CSA) and \emph{Grid Search} are implemented in the optimization module and are used to tune the hyper-parameters of the above model. Finally, we carry out experiments on standard data sets like \emph{Magic Gamma}, \emph{Ripley} and \emph{Forest Cover} and evaluate the performance of various kernel in combination with LS-SVMs.     
\end{abstract}

\maketitle

%%%%%%%%%%%%%%%%%%%%%%%%%%%%%%%%%%%%%%%%%%%%
%% MAINMATTER
%%%%%%%%%%%%%%%%%%%%%%%%%%%%%%%%%%%%%%%%%%%%

\section{Background}

The 21'st century stands out in how mankind learned the value of storing and making predictions/decisions from large volumes of data. A significant aspect of large scale data analysis is distributed computation frameworks like \textit{High Performance Computing}, \textit{Message Passing Interface} etc. Recently large scale commodity hardware clusters have replaced the two former frameworks as the most popular model for parallel data analysis. With this crucial change in hardware came a change in computational models as well. It is at this juncture that distributed \textit{Map Reduce} became the de-facto computational philosophy for large scale data analysis and  words such as \textit{Hadoop} and \textit{Apache Spark} have become synonymous with large scale data analysis and machine learning.

Along with innovation in hardware design and distributed computing models, there came a need for good programming libraries and frameworks to work with various Machine Learning models on large data sets. Halevy et. al \cite{10.1109/MIS.2009.36} in the paper titled "The Unreasonable Effectiveness of Data" demonstrated how gigantic language corpus encapsulate almost all aspects of human language and speech. So far the prevalent 'motto' in the Internet industry has been "large data, simple models", which is a misunderstood phrase as it can be confused as a statement of \textit{Occam's Razor}. The decomposition of the generalization error in Machine Learning is the well known "bias-variance" trade off \cite{Valentini2004}, from that one can observe that there is an optimal point between the bias and variance of a model such that the generalization error is minimum.

Therefore, in order to extract maximum value from large scale data, one must have the flexibility to train and compare different model families before arriving at one that fits the need. This is not possible in a rigid, monolithic programming framework. Modularity, extensibility and ease of use are of paramount importance while designing Machine Learning software for large scale data applications. In the sections that follow, we review the \textit{Fixed Size Least Squares Support Vector Machine} algorithm as outlined by Brabanter et al \cite{DeBrabanter2010} and give an introduction to \textit{Bayes Learn} and its component modules.

\section{Least Squares Support Vector Machines}

\subsection{Formulation}
 Least Squares Support Vector Machines (LS-SVM) \cite{Suykens1999} are a modification of the original SVM formulation by replacing the \textit{hinge} loss function by the \textit{squared error} loss function as shown in equations .
 
\begin{equation}
J_{ion}=A\frac{exp\left[-\frac{E_a}{kT}\right]}{kT}\alpha \label{ionflux}
\end{equation} 

\subsubsection{<A subsubsection>}

Et iam nox umida caelo praecipitat $J_{ion}$ suadentque cadentia
sidera somnos. Sed si tantus amor casus cognoscere nostros et breviter
Troiae $J_{ion}$ supremum audire laborem:
\begin{equation}
J_{ion}=A\frac{exp\left[-\frac{E_a}{kT}\right]}{kT}\alpha \label{ionflux}
\end{equation}
lamentabile regnum cruerint Danai; quaeque ipse miserrima vidi, et
quorum pars magna fui. $A$ talia fando, $E_a$ iam nox umida, $k$ caelo
praecipitat, suadentque cadentia sidera somnos. See \eqref{ionflux}.

\paragraph{<A subsubsubsection>}

Infandum, regina, iubes renovare dolorem, Troianas ut opes et
lamentabile regnum cruerint Danai; quaeque ipse miserrima vidi, et
quorum pars magna fui \cite{Suykens2002}. Quis talia fando
Myrmidonum Dolopumve aut duri miles Ulixi temperet
\cite{Suykens2002} a lacrimis? Et iam
nox umida caelo praecipitat, suadentque \cite{Suykens2002} cadentia
sidera somnos.

%%%%%%%%%%%%%%%%%%%%%%%%%%%%%%%%%%%%%%%%%%%%
%% Sample figure:
%%
%% The option [height=...] scales the picture to the given height,
%% without it it would be printed at its nominal size
%%%%%%%%%%%%%%%%%%%%%%%%%%%%%%%%%%%%%%%%%%%%

\begin{figure}
  \includegraphics[height=.3\textheight]{escher}
  \caption{Picture to fixed height}
\end{figure}

Infandum, regina, iubes renovare dolorem, Troianas ut opes et
lamentabile regnum cruerint Danai; quaeque ipse miserrima vidi, et
quorum pars magna fui. Quis talia fando Myrmidonum Dolopumve aut duri
miles Ulixi temperet a lacrimis?

Infandum, regina, iubes renovare dolorem, Troianas ut opes et
lamentabile regnum cruerint Danai; quaeque ipse miserrima vidi, et
quorum pars magna fui. Quis talia\footnote{A few more footnotes} fando
Myrmidonum Dolopumve aut duri miles Ulixi temperet a lacrimis? Et iam
nox umida caelo praecipitat, suadentque cadentia\footnote{Here we test
footnotes.} sidera somnos. Sed si tantus amor casus cognoscere nostros
et breviter Troiae supremum audire laborem, quamquam animus meminisse
horret, luctuque refugit, incipiam.
In the following we test itemize environments up to the forth level.
\begin{itemize}
\item
  An item with more than a line of text. Infandum, regina, iubes
  renovare dolorem, Troianas ut opes et lamentabile regnum cruerint
  Danai.
\item
  Another item with sub entries
  \begin{itemize}
  \item
   A sub entry.
  \item
   Second sub entry.
    \begin{itemize}
    \item
     A sub sub entry.
      \begin{itemize}
      \item
       A sub sub sub entry.
      \item
       Second sub sub sub entry.
      \end{itemize}
    \item
     Second sub sub entry.
    \end{itemize}
  \end{itemize}
\item
  A final item.
\end{itemize}

%%%%%%%%%%%%%%%%%%%%%%%%%%%%%%%%%%%%%%%%%%%%
%% SAMPLE TABLE
%%
%% Shows the use of \tablehead and \tablenote
%% macros
%%%%%%%%%%%%%%%%%%%%%%%%%%%%%%%%%%%%%%%%%%%%

\begin{table}
\begin{tabular}{lrrrr}
\hline
  & \tablehead{1}{r}{b}{Single\\outlet}
  & \tablehead{1}{r}{b}{Small\tablenote{2-9 retail outlets}\\multiple}
  & \tablehead{1}{r}{b}{Large\\multiple}
  & \tablehead{1}{r}{b}{Total}   \\
\hline
1982 & 98 & 129 & 620    & 847\\
1987 & 138 & 176 & 1000  & 1314\\
1991 & 173 & 248 & 1230  & 1651\\
1998\tablenote{predicted} & 200 & 300 & 1500  & 2000\\
\hline
\end{tabular}
\caption{Average turnover per shop: by type
  of retail organisation}
\label{tab:a}
\end{table}

Infandum, regina, iubes renovare dolorem, Troianas ut opes et
lamentabile regnum cruerint Danai; quaeque ipse miserrima vidi, et
quorum pars magna fui. Quis talia fando Myrmidonum Dolopumve aut duri
miles Ulixi temperet a \cite{Suykens2002} lacrimis? In the following we
test enumrerate environments up to the second level. In addition we
look how ridiculous large labels look.
\begin{enumerate}
\item
  An item \cite{Suykens2002}
\item
  Another item with sub entries
  \begin{enumerate}
  \item
   A sub entry \cite{Suykens2002}
  \item
   Second sub entry
  \end{enumerate}
\item
  The final item with normal label.
\end{enumerate}
Infandum, regina, iubes renovare dolorem, Troianas ut opes et
lamentabile regnum cruerint Danai; quaeque ipse miserrima vidi, et
quorum pars magna fui. Quis talia  fando Myrmidonum Dolopumve aut duri
miles Ulixi temperet a lacrimis?
\begin{description}
\item[Infandum]
 regina, iubes renovare dolorem, Troianas ut opes et lamentabile
 regnum cruerint Danai.
\item[Sed]
 si tantus amor casus cognoscere nostros et breviter Troiae supremum
 audire laborem, quamquam animus meminisse horret, luctuque refugit,
 incipiam.
\item[Lamentabile] regnum cruerint Danai; quaeque ipse miserrima vidi, et
quorum pars magna fui. Quis talia  fando Myrmidonum Dolopumve aut duri
miles Ulixi temperet a lacrimis?
\end{description}

Infandum, regina, iubes renovare dolorem, Troianas ut opes et
lamentabile regnum cruerint Danai; quaeque ipse miserrima vidi, et
quorum pars magna fui. Quis talia fando Myrmidonum Dolopumve aut duri
miles Ulixi temperet a lacrimis?
Infandum, regina, iubes renovare dolorem, Troianas ut opes et
lamentabile regnum cruerint Danai; quaeque ipse miserrima vidi, et
quorum pars magna fui. Quis talia fando Myrmidonum Dolopumve aut duri
miles Ulixi temperet a lacrimis?

Infandum, regina, iubes renovare dolorem, Troianas ut opes et
lamentabile regnum cruerint Danai; quaeque ipse miserrima vidi, et
quorum pars magna fui. Quis talia fando Myrmidonum Dolopumve aut duri
miles Ulixi temperet a lacrimis? Et iam nox umida caelo praecipitat,
suadentque cadentia sidera somnos. Sed si tantus amor casus
\cite{Suykens2002} cognoscere nostros et breviter Troiae supremum
audire laborem, quamquam animus meminisse horret, luctuque refugit,
incipiam.  Infandum, regina, iubes renovare dolorem, Troianas ut opes
et lamentabile regnum cruerint Danai; quaeque ipse miserrima vidi, et
quorum pars magna fui. Quis talia fando Myrmidonum Dolopumve aut duri
miles Ulixi temperet a \cite{DeBrabanter2010} lacrimis? Et iam nox umida caelo
praecipitat, suadentque cadentia sidera somnos. Sed si tantus amor
casus cognoscere nostros et breviter Troiae supremum audire laborem,
quamquam animus meminisse horret, luctuque refugit, incipiam.

\section{<A section>}

Infandum, regina, iubes renovare dolorem, Troianas ut opes et
lamentabile regnum cruerint Danai; quaeque ipse miserrima vidi, et
quorum pars magna fui. Quis talia fando Myrmidonum Dolopumve aut duri
miles Ulixi temperet a lacrimis?

Et iam nox umida caelo praecipitat, suadentque cadentia sidera
somnos. Sed si tantus amor casus cognoscere nostros et breviter Troiae
supremum audire \cite{Suykens2002} laborem, quamquam animus meminisse
horret, luctuque refugitum, refugit, incipitat, suadenovare dolorem,
Troianas ut opes Ulixi temperet breviter Troiaeque ipse nostros et a
lacrimis?

Infandum, regina, iubes renovare dolorem, Troianas ut opes et
lamentabile regnum cruerint \cite{Brabanter2011} Danai; quaeque ipse
miserrima vidi, et quorum pars magna fui. Quis talia fando Myrmidonum
Dolopumve aut duri miles Ulixi temperet a lacrimis?  Infandum, regina,
iubes renovare dolorem, Troianas ut opes et lamentabile regnum
cruerint Danai; quaeque ipse miserrima vidi, et quorum pars magna
fui. Quis talia fando Myrmidonum Dolopumve aut duri miles Ulixi
temperet a lacrimis?

%%%%%%%%%%%%%%%%%%%%%%%%%%%%%%%%%%%%%%%%%%%%%%%%
%% BACKMATTER
%%%%%%%%%%%%%%%%%%%%%%%%%%%%%%%%%%%%%%%%%%%%%%%%

\begin{theacknowledgments}
  Infandum, regina, iubes renovare dolorem, Troianas ut opes et
  lamentabile regnum cruerint Danai; quaeque ipse miserrima vidi, et
  quorum pars magna fui. Quis talia fando Myrmidonum Dolopumve aut duri
  miles Ulixi temperet a lacrimis?
\end{theacknowledgments}

%%%%%%%%%%%%%%%%%%%%%%%%%%%%%%%%%%%%%%%%%%%%%%%%
%% The bibliography can be prepared using the BibTeX program or
%% manually.
%%
%% The code below assumes that BibTeX is used.  If the bibliography is
%% produced without BibTeX comment out the following lines and see the
%% aipguide.pdf for further information.
%%
%% For your convenience a manually coded example is appended
%% after the \end{document}
%%%%%%%%%%%%%%%%%%%%%%%%%%%%%%%%%%%%%%%%%%%%%%%%

%%%%%%%%%%%%%%%%%%%%%%%%%%%%%%%%%%%%%%%%%%%%%%%%
%% You may have to change the BibTeX style below, depending on your
%% setup or preferences.
%%
%%
%% For The AIP proceedings layouts use either
%%%%%%%%%%%%%%%%%%%%%%%%%%%%%%%%%%%%%%%%%%%%

\bibliographystyle{aipproc}   % if natbib is available
%\bibliographystyle{aipprocl} % if natbib is missing

%%%%%%%%%%%%%%%%%%%%%%%%%%%%%%%%%%%%%%%%%%%
%% You probably want to use your own bibtex database here
%%%%%%%%%%%%%%%%%%%%%%%%%%%%%%%%%%%%%%%%%%%
\bibliography{sample}

%%%%%%%%%%%%%%%%%%%%%%%%%%%%%%%%%%%%%%%%%%%
%% Just a reminder that you may have to run bibtex
%% All of it up to \end{document} can be removed
%% if you don't like the warning.
%%%%%%%%%%%%%%%%%%%%%%%%%%%%%%%%%%%%%%%%%%%
\IfFileExists{\jobname.bbl}{}
 {\typeout{}
  \typeout{******************************************}
  \typeout{** Please run "bibtex \jobname" to optain}
  \typeout{** the bibliography and then re-run LaTeX}
  \typeout{** twice to fix the references!}
  \typeout{******************************************}
  \typeout{}
 }

\end{document}
\endinput
%%
%% End of file `template-8d.tex'.
